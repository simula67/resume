\documentclass[line,margin]{res}

\usepackage{color}

\begin{document}
\name{Joji Antony}

\address{jojiantony@yandex.com}
\address{+91 96323 41954}

\begin{resume}

\section{Profile}
Back end leaning full stack developer with 8+ years of experience in the IT industry, revolving around designing, building, testing and implementing infrastructure software solutions for clients in diverse industries.

\section{Education}
{\bf Undergraduate College Education}\\
Bachelor of Technology in Computer Science and Engineering from College of Engineering, Trivandrum\\

\section{Technical Skills}

{\bf Programming Languages}\\
Go, Python, C, C++, Java (EE 6, Hibernate), JavaScript (React, Redux), Perl etc.

{\bf Platforms}\\
Linux, Amazon Web Services, Kubernetes, Heroku, MySQL, Postgres, Docker etc.

\section{Work Experience and Projects}

Employed at Arista Networks since April 2018\\

{\bf Arista CloudVision Portal} \hfill {\it{2018 - 2020}}\\

CloudVision is a network management platform which is based on Arista Network Database. Arista EOS is a Linux based Network OS that is capable of being extended with an application that can stream the current switch state to a network management platform to produce a network database (NetDB). This extension can be used to provision configuration, stream telemetry, run scripts etc as directed by the network management platform. 

{\it{Accomplishments}}\\
\begin{itemize}
\item Design and implement Kubernetes based container farm for scale testing
\item Design and implement a Sphinx based course for on boarding of new members 
\item Implement scale tests with reporting features using InfluxDB
\item Improve product scale to support more devices
\item Develop automated tests for new and existing features
\item Provide support to customers during on-call rotation
\end{itemize}
{\bf Nutanix Move} \hfill {\it{2016 - 2018}}\\

Nutanix Move is a migration platform to migrate workloads from legacy infrastructure to Nutanix Cloud. It consists of Xtract for migrations and Xplorer for reconnaissance.

Nutanix Cloud is a scale out hyper-converged private cloud platform from the authors of Google File System that unifies storage, compute and networking into a single pane of management glass. 

{\it{Accomplishments}}\\
\begin{itemize}
\item Lead the AHV hypervisor support feature in Nutanix Xtract for DBs
\item Design and implement error handling and remediation
\item Lead the release of Nutanix Xplorer 1.0
\item Design and implement static IP address deployment using Windows sysprep
\item Design and implement ISO file verification using Nutanix NFS support
\end{itemize}

{\bf HPE Hyperconverged cloud} \hfill {\it{2016 - 2016}}\\

HPE Hyperconverged was a private cloud platform project based on HPE StoreVirtual storage aggregator software. As a front end developer, my role was to work with the user experience team to implement the user interface in React and Redux.

{\it{Accomplishments}}\\
Implemented Image Management UI\\

{\bf HP Network Node Manager} \hfill {\it{2013 - 2016}}\\

HP Network Node Manager (NNM) is a collection of software that helps users analyze, manage and monitor their networks from the designers of SNMP. The platform discovers network topology, stitches them into a graph, polls the devices periodically and reports root cause errors using an analyser engine. The Smart Plugin (SPI) architecture allows the platform to be extended to provide integration to other automation platforms for remediation, run book automation among other features.

{\it{Accomplishments}}\\
\begin{itemize}
\item Improve SNMP v3 USM specification compliance in the SNMP stack
\item Enable Network Function Virtualization support
\item Port the Network Engineer's Tool set to Linux
\item Add User Interface support for Safari on Mac
\item Add CSRF protection to the NNM
\item Release NNM SPI NET 10.0, 10.10, 10.20, NNM 10.10, NNM 10.20, and many other patchsets
\item Solve many multi-threading issues (ACE framework)
\item Solve many cache coherence issues
\item Support NNM on 6 platforms (Linux 2.4 x86, Linux 2.6 x86, Solaris SPARC, Windows x86, HP-UX Itanium and HP-UX PARISC)
\end{itemize}


\section{Side projects}
{\bf Plivo Call Center}\\
Functional call center application using Plivo API running on the Heroku Platform
https://github.com/simula67/plivo-callcenter

\section{Online Profiles}
{\bf LinkedIn :} \it{www.linkedin.com/in/joji-antony} \\
{\bf Github :} \it{https://github.com/simula67/}

\end{resume}
\end{document}
