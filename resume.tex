\documentclass[line,margin]{res}

\usepackage{color}

\begin{document}
\name{Joji Antony}


\address{jojiantony@yandex.com}
\address{+91 96323 41954}

\begin{resume}

\section{Profile}
Back end leaning full stack developer with 8+ years of experience in the IT industry, revolving around designing, building, testing and implementing infrastructure software solutions for clients in diverse industries.

\section{Education}
{\bf Undergraduate College Education}\\
Bachelor of Technology in Computer Science and Engineering from College of Engineering, Trivandrum\\

\section{Technical Skills}

{\bf Technologies}\\
Go, Python, C, C++, Java (EE 6, Hibernate), JavaScript (React, Redux), Perl, Docker, Kubernetes, MySQL, Postgres etc

{\bf Platforms}\\
Amazon Web Services, Heroku

\section{Work Experience and Projects}

Employed at Arista Networks since April 2018\\

{\bf Arista CloudVision Portal} \hfill {\it{2018 - Present}}\\

CloudVision is a network management platform which is based on Arista Network Database. Arista EOS is a Linux based Network OS that is capable of being extended with an application which can stream the current switch state to a network management platform and form a network database (NetDB). This extension can be used to provision configuration, stream telemetry, run scripts etc as directed by the network management software (CloudVision). 

{\it{Accomplishments}}\\
\begin{itemize}
\item Setting up a Kubernetes based container farm for devices for scale testing
\item Coordinating and facilitating the quick on boarding of new members by using a Sphinx based course
\item Creating scale tests with reporting features using InfluxDB
\item Improving product scale to support more devices
\item Developing automated tests for new and existing features
\item Providing support to customers during on-call rotation
\end{itemize}
{\bf Nutanix Move} \hfill {\it{2016 - 2018}}\\

Nutanix Move is a migration platform to migrate workloads from legacy infrastructure to Nutanix Cloud. It consists of Xtract for migrations and Xplorer for reconnaissance.

Nutanix Cloud is a hyper-converged private cloud platform from the authors of Google File System that unifies storage, compute and networking into a single pane of management glass and a scale out design. 

{\it{Accomplishments}}\\
\begin{itemize}
\item Lead the AHV hypervisor support feature in Nutanix Xtract for DBs
\item Implemented error handling and remediation of Nutanix Xplorer
\item Lead the release of Nutanix Xplorer 1.0.
\end{itemize}


{\bf HPE Hyperconverged cloud} \hfill {\it{2016 - 2016}}\\

HPE Hyperconverged was a private cloud platform project based on HPE StoreVirtual storage aggregator software. On this project as a front end developer, my role was to work with the user experience team to implement the user interface in React and Redux.

{\it{Accomplishments}}\\
Implemented Image Management UI\\

{\bf HP Network Management Center} \hfill {\it{2013 - 2016}}\\
HP Network Management Center is a collection of software that helps users analyze, manage and monitor their networks from the designers of SNMP. The software discovers network topology, stitches them into a graph, polls the devices periodically and reports root causes using an analyser engine. The software also provides integration to other automation platforms for remediation and run book automation among other features.

{\it{Accomplishments}}\\
\begin{itemize}
\item Improving SNMP v3 USM specification compliance in the SNMP stack
\item Enabling Network Function Virtualization support
\item Port the Network Engineer's Tool set to Linux
\item Add UI support for Safari on Mac
\item Adding CSRF protection to the NNMi
\item Released NNMi SPI NET 10.0, 10.10, 10.20, NNMi 10.10 and NNMi 10.20
\item Solved many multi-threading issues (ACE framework)
\item Solved many cache coherence issues
\item Released NNM 7.53 Intermediate Patches 31 and 32
\end{itemize}


\section{Side projects}
{\bf Plivo Call Center} \hfill \it{2015-Present}\\
Functional call center application using Plivo API running on the Heroku Platform
https://github.com/simula67/plivo-callcenter\\

\section{Online Profiles}
{\bf LinkedIn :} \it{www.linkedin.com/in/joji-antony} \\
{\bf Github :} \it{https://github.com/simula67/} \\

\end{resume}
\end{document}
