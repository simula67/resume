\documentclass[line,margin]{res}

\usepackage{color}

\begin{document}
\name{Joji Antony}


\address{jojiantony@yandex.com}
\address{+91 96323 41954}

\begin{resume}

\section{Profile}
Back end leaning full stack developer with over eight years of experience in the IT industry, revolving around designing, building, testing and implementing infrastructure software solutions for clients in diverse industries.

\section{Education}
{\bf Undergraduate College Education}\\
Bachelor of Technology in Computer Science and Engineering, College of Engineering, Trivandrum\\

\section{Technical Skills}

{\bf Technologies}\\
Go, Python, C, C++, Java (EE, Hibernate), Javascript (React, Redux), Perl, 
Docker, Kubernetes ,Postgres etc

{\bf Platforms}\\
Amazon Web Services, Heroku

\section{Work Experience and Projects}

Employed at Arista Networks since April 2018\\

{\bf Arista CloudVision Portal} \hfill {\it{2018 - Present}}\\

CloudVision is a network management software which is based on Arista network database. This works by installing an extension on the network switches that streams data from the switch and is used to provision, manage and analyze the network.

{\it{Accomplishments}}\\
\begin{itemize}
\item Setting up a Kubernetes based container farm for devices for testing scale
\item Coordinating and facilitating the quick on boarding of new members by using a Sphinx based course
\item Creating scale tests with reporting features using InfluxDB
\item Improving product scale to support more devices
\item Developing automated tests
\item Providing support to customers during on-call rotation
\end{itemize}

{\bf Nutanix Move} \hfill {\it{2016 - 2018}}\\

Nutanix Move is a migration platform to migrate workloads from legacy infrastructure to Nutanix private cloud. It consists of Xtract for migrations and Xplorer for reconnaissance.

Nutanix private cloud is a hyper-converged private cloud platform that unifies storage, compute and networking into a single pane of management glass with a scale out design. 

{\it{Accomplishments}}\\
\begin{itemize}
\item Lead the AHV hypervisor support feature in Nutanix Xtract for DBs
\item Implemented error handling and remediation of Nutanix Xplorer
\item Lead the release of AMF Xplorer 1.0.
\end{itemize}


{\bf HPE Hyperconverged cloud} \hfill {\it{2016 - 2016}}\\

HPE Hyperconverged was a private cloud platform project based on HPE StoreVirtual storage aggregator software. On this project as a front end developer, my role was to work with the user experience team to implement the user interface in React and Redux.

{\it{Accomplishments}}\\
Implemented Image Management UI\\

{\bf HP Network Management Center} \hfill {\it{2013 - 2016}}\\
HP Network Management Center is a collection of software that helps users analyze, manage and monitor their networks.

{\it{Accomplishments}}\\
\begin{itemize}
\item Improving SNMP v3 USM specification compliance in the SNMP stack
\item Enabling Network Function Virtualization support
\item Port the Network Engineer's Tool set to Linux
\item Add UI support for Safari on Mac
\item Adding CSRF protection to the NNMi
\item Released NNMi SPI NET 10.0, 10.10, 10.20, NNMi 10.10 and NNMi 10.20
\end{itemize}


{\bf HP Network Node Manager 7.x lab support} \hfill {\it{2011 - 2013}}\\
Lab support engineer for Hewlett-Packard's premier Network Management Solution called NNM\\


{\it{Accomplishments}}\\
\begin{itemize}
\item Solved many multi-threading issues (ACE framework)
\item Solved many cache coherence issues
\item Released NNM 7.53 Intermediate Patches 31 and 32
\end{itemize}


\section{Side projects}
{\bf Plivo Call Center} \hfill \it{2015-Present}\\
Functional call center application using Plivo API running on the Heroku Platform
https://github.com/simula67/plivo-callcenter\\

\section{Online Profiles}
{\bf LinkedIn :} \it{www.linkedin.com/in/joji-antony} \\
{\bf Github :} \it{https://github.com/simula67/} \\

\end{resume}
\end{document}
